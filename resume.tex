\documentclass{resume} % Use the custom resume.cls style

\usepackage[left=0.3 in,top=0.3in,right=0.3 in,bottom=0.3in]{geometry} % Document margins

\newcommand{\tab}[1]{\hspace{.2667\textwidth}\rlap{#1}} 
\newcommand{\itab}[1]{\hspace{0em}\rlap{#1}}

\name{Jackson Isenberg} % Your name

\address{770-668-6875 \\ \href{mailto:jisenberg3@gatech.edu}{jisenberg3@gatech.edu} \\ \href{https://www.github.com/JIceberg}{github.com/JIceberg} \\ \href{https://www.linkedin.com/in/jaxonfiles}{linkedin.com/in/jaxonfiles}}  % Your phone number, email, linkedin, and (optional) website

\begin{document}

%----------------------------------------------------------------------------------------
%	EDUCATION SECTION
%----------------------------------------------------------------------------------------

\vspace{-0.5em}
\begin{rSection}{Education}

{\bf Georgia Institute of Technology} \hfill {August 2024 -- Present}\\
Master of Science in Robotics

{\bf Georgia Institute of Technology} \hfill {June 2020 -- May 2024}\\
Bachelor of Science in Computer Science, Minor in Robotics
\vspace{-0.5em}
\begin{itemize}
   \itemsep -5pt {}
   \item Concentrations: Systems \& Architecture, Artificial Intelligence and Machine Learning
   \item Coursework: Linear Control Theory, Control System Design, Deep Learning, Embedded Programming, Operating Systems, Compilers, Computer Vision,
   Signal Processing, Data Structures, Algorithms, AI for Robotics, Probability \& Statistics
\end{itemize}

\end{rSection}

%----------------------------------------------------------------------------------------
%	WORK EXPERIENCE SECTION
%----------------------------------------------------------------------------------------

\vspace{-0.8em}
\begin{rSection}{EXPERIENCE}
\textbf{Low-power, Adaptive, and Resilient Systems Lab} \hfill Atlanta, GA\\
\textit{Undergraduate Research Assistant} \hfill August 2022 -- May 2024
\vspace{-0.5em}
   \begin{itemize}
      \itemsep -5pt {} 
      \item Worked with the Amazon AWS Deepracer stack to locally train and test a DNN-based RL model
      for autonomous vehicle pathing utilizing environment information from camera input
      \item Created an architecture-agnostic fault injection and resilience framework in TensorFlow
   \end{itemize}
\vspace{-0.5em}
\textbf{Georgia Tech Research Institute} \hfill Atlanta, GA\\
\textit{Student Research Assistant (TMPO Lab, CIPHER)} \hfill May 2021 -- May 2024
\vspace{-0.5em}
 \begin{itemize}
    \itemsep -5pt {} 
     \item Designed and implemented the first real-time operating system in Rust for the Cortex R4 where
     nearly 100\% of Rust's safety features at abstraction levels above the bootloader were utilized to improve
     upon the critical safety of the system
     \item Worked on various FPGA projects related to architecture analysis and bitstream generation (secret clearance)
 \end{itemize}
\textit{Research Intern (ATAS)} \hfill June -- July 2020
\vspace{-0.5em}
 \begin{itemize}
    \itemsep -5pt {} 
     \item Worked and modeled a 5 degree-of-freedom Arduino-powered arm and
     developed a C++ library for the arm's inverse kinematics
     \item Researched various OpenCV-extendable libraries such as AprilTags for detecting visual orientation of the end effector
 \end{itemize}
\textit{Research Intern (ATAS)} \hfill June -- July 2019
\vspace{-0.5em}
 \begin{itemize}
    \itemsep -5pt {} 
     \item Improved the design of the battery compartments in the piezoelectric tiles found at the Kennedy Space Center
     to prevent expansion over time due to trapped heat
     \item Sole researcher of liquid treatment using UV-C LEDs for the Gates Foundation
     Reinvented Toilet which had an effective wavelength range of 250-300 nm
 \end{itemize}
\end{rSection} 

%----------------------------------------------------------------------------------------
% TECHINICAL STRENGTHS	
%----------------------------------------------------------------------------------------

\vspace{-0.8em}
\begin{rSection}{PROJECTS}
\vspace{-1.25em}
\item \textbf{Neuraphonic} (HackGT X) | \textit{Python, PyTorch, Scikit-Learn, Google Cloud, Twilio, MATLAB} \\
Developed a diagnostic assistant for Parkinson's disease utilizing a vision transformer and signal processing techniques to extract
various features from audio samples uploaded to a website hosted on Google Cloud or through telephone. Won 2nd place best overall project
out of 189 total teams.
\vspace{-0.5em}
\item \textbf{FTCLib} | \textit{Java, Kotlin, OpenCV}\\
Founded and led the development of a Java library for FIRST Tech Challenge used by hundreds of teams internationally
to enhance their software efficiency and experience.
\vspace{-0.5em}
\item \textbf{Grouch} | \textit{Python} \\
Created a scraping program in Python 3 that aids the registration process for Georgia Tech students by
checking vacant spots and available waitlists as an alternative to the currently paid service that students use.
\end{rSection}

\vspace{-0.8em}
\begin{rSection}{EXTRACURRICULAR}
\vspace{-1.25em}
\item \textbf{HyTech Racing}\\
\textit{Data Acquisition}
\vspace{-0.5em}
 \begin{itemize}
    \itemsep -5pt {} 
    \item Designed schematics and fabricated PCBs to retrieve sensor data to be analyzed
    \item Programmed and tested Arduino/Teensy microcontrollers over a CAN line for
    messages containing sensor data to be parsed into a useful, readable format for debugging and testing
 \end{itemize}
\vspace{-0.5em}
\item \textbf{RoboJackets}\\
\textit{IT Coordinator}
\vspace{-0.5em}
 \begin{itemize}
    \itemsep -5pt {} 
    \item Managed the networks and distributed services provided to over 600 members, including setting up mailing lists and
    the shared file system used by team members
    \item Provided assistance to any members experiencing issues with their provided services, connections, or loaned devices
 \end{itemize}
\end{rSection}

\vspace{-0.75em}
\begin{rSection}{SKILLS}
\begin{tabular}{ @{} >{\bfseries}l @{\hspace{6ex}} l }
Languages & Java, Python, C/C++, Rust, Verilog, VHDL, HTML, JavaScript, MATLAB \\
Frameworks & NumPy, PyTorch, TensorFlow, JavaFX, React \\
Software & Git, AWS, Docker, ROS/ROS2, Virtual Machines \\
\end{tabular}
\end{rSection}
\end{document}

